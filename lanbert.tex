\documentclass{article}
\usepackage{amsmath}
\usepackage{siunitx}

\begin{document}

\title{The Beer-Lambert Law}
\author{}
\date{}
\maketitle

\section*{The Beer-Lambert Law}

The Beer-Lambert Law relates the attenuation of light to the properties of the material through which the light is traveling. The law is usually written as:

\begin{equation}
A = \epsilon c l
\end{equation}

where:
\begin{itemize}
    \item $A$ is the absorbance (no units, as it is a logarithmic ratio),
    \item $\epsilon$ is the molar absorptivity or molar extinction coefficient (in units of \si{\per\mole\per\centi\meter}),
    \item $c$ is the concentration of the absorbing species in the solution (in units of \si{\mole\per\liter}),
    \item $l$ is the path length through which the light passes (in units of \si{\centi\meter}).
\end{itemize}

The absorbance $A$ is defined as:

\begin{equation}
A = -\log_{10}\left(\frac{I}{I_0}\right)
\end{equation}

where:
\begin{itemize}
    \item $I_0$ is the incident light intensity,
    \item $I$ is the transmitted light intensity.
\end{itemize}

\end{document}
